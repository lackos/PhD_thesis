\documentclass[10pt,a4paper]{report}
\usepackage[utf8]{inputenc}
\usepackage[english]{babel}
\usepackage{amsmath}
\usepackage{amsfonts}
\usepackage{amssymb}
\usepackage{graphicx}
\usepackage{setspace}
\usepackage[left=2cm,right=2cm,top=2cm,bottom=2cm]{geometry}
\begin{document}
\doublespacing
\title{PhD Work}
\date{\today}
\author{Bryce Lackenby}

\maketitle
\chapter{Background Information}
\section{Nuclear Units}
In this thesis we will use the framework of nuclear units. In this framework we set $\hbar = c = 1$. 
\begin{itemize}
\item included cited discussion on the concept of using low energy detection to detect high energy phenomena.
\end{itemize}
\section{Parity nonconservation in atomic systems}
The standard model has stood the test of time coming out on top time again after countless tests. Detection of the violation of fundamental symmetries is an avenue which is suspected to overthrow the model by suggesting a deeper underlyng theory. It is suspected that the standard model is a low energy approximation to a more fundamental theory of the universe similar to how classical physics is a large spacial approximation to quantum theory. There are a number of discrete symmetries which form the basis of our current interpretation of the universe. They are time-reversal $T$ parity and cahrge conjugation (get the paragraph from Honours thesis). \\

Of these discrete symmetries only parity has been determined to be violated. This is a prediction of the standard model weak interaction and therefore the bubble has yet to burst on the standard model. In this thesis we will look at some processes which violate these symmetries. 
\section{CPT Theory}
\chapter{Weak Quadrupole Moment}
\section{Neutron Quadrupole moment}
The deviation of density from a spherical symmetry is known as the quadrupole moment. The spectroscopic moment along the $z$-axis of the nucleus is given by,
\begin{align*}
Q_{zz} = e \sum{N}\left(2z^2 - x^2 -y^2\right)
\end{align*}
where specifically the electric quadrupole moment which is the cahrge deviation from spherical symmetry is given by,
\begin{align*}
Q_{e} = eQ_{zz}
\end{align*}
\begin{itemize}
\item Compared to proton quadrupole moment.
\item include mathematical descritpion including charge and density.
\item Discuss connection to Weak charge and PNC
\item 
\end{itemize}
\chapter{MQM in Nilsson Model}
\section{Splitting of energy bands in deformed nuclei}
In the Nilsson model the energy levels are split due to the 
\section{Lorentz violating parameters in nuclei}
\section{Quadrupole moments of neutrons in nuclei}
\section{Magnetic quadrupole moment in deformed nuclei}
TP violating nuclear moments induce a spin hedgehog wavefunction.
\begin{align*}
\left|\psi'\right> = \left(1 + \xi\hat{\boldsymbol{\sigma}}\hat{\boldsymbol{\nabla}}\right)\left|\psi_0\right>
\end{align*}
where $\left|\psi_0\right>$ is the unperturbed wavefunction. The magnetic quadrupole moment of the nucleus (MQM) is defined by the second order tensor operator,
\begin{align*}
\hat{M}_{kn} = \dfrac{e}{2m}\left[3\mu\left(r_k\sigma_n + \sigma_kr_n - \dfrac{2}{3}\delta_{kn}\hat{\boldsymbol{\sigma}}\textbf{r}\right) \right. \\
\left. + 2q\left(r_kl_n + l_kr_n\right)\right]
\end{align*}
for the unperturbed wavefunction the matrix element vanishes $\left<\psi_0\right|\hat{M_{kn}}\left|\psi_0\right> = 0$. However after it is perturbed by the TP odd interaction there is a non zero matrix element.
\begin{align*}
M_{kn} &= \left<\psi'\right|\hat{M_{kn}}\left|\psi'\right> \\
&= \left<\psi_0\right|\hat{M_{kn}}\left|\psi_0\right> \\
-\xi  \left<\psi_0\right|\left[\hat{\boldsymbol{\sigma}}\hat{\boldsymbol{\nabla}}, \hat{M}_{kn}\right]\left|\psi_0\right> \\ - \xi^{2} \left<\psi_0\right|\hat{\boldsymbol{\sigma}}\hat{\boldsymbol{\nabla}}\hat{M}_{kn}\hat{\boldsymbol{\sigma}}\hat{\boldsymbol{\nabla}}\left|\psi_0\right>
\end{align*}
as stated above the first term vanishes also ignore the last term as it is heavily suppressed due to $\xi^2$ the MQM due to the TP violating effects is given by the matrix element,
\begin{align*}
M_{kn} = -\xi  \left<\psi_0\right|\left[\sigma_{\nu}\nabla_{\nu}, \hat{M}_{kn}\right]\left|\psi_0\right>
\end{align*}
This simplifies to 4 effective matrix elements,
\begin{align*}
M^{(1)}_{kn} &= \left<\psi_0\right|\sigma_m\nabla_m\left(r_k\sigma_n + \sigma_kr_n - \dfrac{2}{3}\delta_{kn}\sigma_{\nu}r_{\nu}\right)\left|\psi_0\right> \\
M^{(2)}_{kn} &= \left<\psi_0\right|\sigma_m\nabla_m\left(r_kl_n + l_kr_n\right)\left|\psi_0\right> \\
M^{(3)}_{kn} &=  \left<\psi_0\right|\left[\sigma_m , r_k\sigma_n + \sigma_kr_n -\dfrac{2}{3}\delta_{kn}\sigma_{\nu}r_{\nu}\right]\nabla_m\left|\psi_0\right> \\
M^{(4)}_{kn} &= \left<\psi_0\right|\left[\sigma_m, r_kl_n + l_kr_n\right]\nabla_m\left|\psi_0\right>
\end{align*}
These 4 matrix elements will determine the MQM in nuclei. We will go through each of the matrix elements separately. The first element,
\begin{align*}
M^{(1)}_{kn} &= \left<\sigma_m\nabla_m\left(r_k\sigma_n + \sigma_kr_n - \dfrac{2}{3}\delta_{kn}\sigma_\nu r_\nu\right)\right> \\
&= \left<\sigma_m\left(\delta_{km}\sigma_{n} + \delta_{mn}\sigma_k - \dfrac{2}{3}\delta_{kn}\delta_{m\nu}\sigma_{\nu}\right)\right> \\
&= \left<\sigma_k\sigma_n + \sigma_n\sigma_k - \dfrac{2}{3}\sigma_{\nu}\sigma_{\nu} \right>
\end{align*}
Using the Pauli matrices properties for the anticommutator and product we have $\left\{\sigma_k, \sigma_n\right\} = 2I_{2}\delta_{kn}$ and $\sigma_{\nu}\sigma_{\nu} = 3I_{2}$ therefore we have that,
\begin{align*}
M^{(1)}_{kn} = 2I_{2}\delta_{kn} - \dfrac{2}{3} 3I_{2} = 0.
\end{align*}
The second matrix element is given by ,
\begin{align*}
M^{(2)}_{kn} &= \left<\sigma_m\delta_m\left(r_kl_n + l_kr_n\right)\right> \\
 &= \left<\sigma_m \left[\delta_{km}l_n + \delta_{nm}l_k\right]\right> \\
&= \left<\sigma_k l_n + \sigma_nl_k\right>
\end{align*}
The third matrix element is given by,
\begin{align*}
M^{(3)}_{kn} &= \left<\left[\sigma_m, r_kl_n + l_kr_n\right]\nabla_m\right> \\
\end{align*}
as $r_i$, $l_{j}$ and $\sigma_k$ commute for $i,j,k$ we have that,
\begin{align*}
M^{(3)}_{kn} = 0.
\end{align*}
For the last matrix element we have,
\begin{align*}
M^{(4)}_{kn} &= \left<\left[\sigma_m, r_k\sigma_n + \sigma_kr_n - \dfrac{2}{3}\delta_{kn}\sigma_{\nu}r_{\nu}\right] \nabla_m\right> \\
&= \left< \left[\sigma_m,r_k\right]]\sigma_n\nabla_m + r_k\left[\sigma_m,\sigma_n\right]\nabla_m + \left[\sigma_m,\sigma_k\right]r_n\nabla_m \right. \\ 
&\left. + \sigma_k\left[\sigma_m,r_n\right]\nabla_m - \dfrac{2}{3}\delta_{kn}\left[\sigma_m,\sigma_{\nu}\right]r_{\nu}\nabla_m\right> \\
&= r_k\left[\sigma_m,\sigma_n\right]\nabla_m + r_n \left[\sigma_m,\sigma_k\right]\nabla_m - \dfrac{2}{3}\delta_{kn}\left[\sigma_m,\sigma_{\nu}\right]r_{\nu}\nabla_{m}
\end{align*}
Using the Pauli commutation relations $\left[\sigma_i\sigma_j\right] = 2i\epsilon_{ijk}\sigma_k$ we have that,
\begin{align*}
M^{(4)}_{kn} = \left<2ir_k\epsilon_{mna}\sigma_a\nabla_m + 2ir_n\epsilon_{mka}\sigma_a\nabla_m\right> - \dfrac{2}{3}\delta_{kn}2i\epsilon_{m\nu a}\sigma_{a}r_{\nu}\nabla_{m}
\end{align*}
We can rewrite this as,
\begin{align*}
M^{(4)}_{kn} = r_kB_n + r_nB_k - \dfrac{2}{3}\delta_{kn}r_{\nu}B_{\nu}
\end{align*}
where $B_{i} = 2i\epsilon_{mia}\sigma_a\nabla_m$. we are interested in the projection of the MQM on the symmetry axis of the nucleus which we will denote as the $z$ axis. Therefore we want to find $M_{zz}$ projection. For the non vanishing matrix elements we will find the value of the projection,
\begin{align*}
M_{zz}^{(2)} = \left<2\sigma_zl_z\right>
\end{align*}
Pauli matrices are related to the spin matrices by the realation,
\begin{align*}
\sigma_i = 2s_i
\end{align*}
and therefore the matrix element is given by,
\begin{align*}
M_{zz}^{(2)} = 4\left<s_zl_z\right>
\end{align*}
For the unperturbed state we will use the Nilsson basis described in the 1955 Nilsson paper \cite{Nilsson1955} $\left|\psi_0\right> = \left|Nl\Lambda\Sigma\right>$. From \cite{Nilsson1955} the matrix element $\left<s_zl_z\right> = \left<s_z\right>\left<l_Z\right> = \Sigma\Lambda$. \\
For the other matrix element we have to simplify the matrix element first. \\

\begin{align*}
\dfrac{\left<r_{\nu}B_{\nu}\right>}{2i} &= \left<r_{\nu}\epsilon_{m\nu a}\sigma_a \nabla_m\right> \\
&= -i\left<\sigma_a\epsilon_{a\nu m}r_{\nu} p_{m} \right> \\
&= -i\left<\boldsymbol{\sigma}\cdot\left(\textbf{r} \times \textbf{p}\right)\right> \\
&= -i\left<\boldsymbol{\sigma}\cdot\textbf{l}\right> \\ 
\Rightarrow \left<r_{\nu}B_{\nu}\right> &= 2\left<\boldsymbol{\sigma}\cdot\textbf{l}\right> \\
&= 4\left<\textbf{s}\cdot\textbf{l}\right> \\
&= 4\Sigma\Lambda
\end{align*}
Also
\begin{align*}
\end{align*}
Therefore the total quadrupole moment is given by,
\section{Schiff moment in nuclei}
\chapter{Superheavy Elements}
\appendix
\section{Second order tensors between rotating frames}
\end{document}